\section{Discussion}

Graph convolutional networks are typically viewed as methods to aggregate information from multiple distances (e.g., first neighbors, second neighbors, etc.).  However, in contrast with images that are typically overlaid on a 2D lattice, graphs have a complex topology. This topology is highly informative of the properties of nodes and edges \cite{rosen2016topological}\cite{naaman2019edge}, and can thus be used to classify their classes.
In undirected graphs, the topology can often be captured by a distance maintaining projection into $\mathbb{R}^{N}$. Such a projection is often obtained by MDS methods \cite{kruskal1964multidimensional}, or using supervised methods to minimize the distance between nodes with similar classes in the training set \cite{cao2016deep}. In directed graphs, a more complex topology emerges from the asymmetry between incoming and outgoing edges (i.e., the distance between node $i$ and node $j$ differs from the distance between node $j$ and node $i$), creating a distribution of subgraphs around each node often denoted sub-graph motifs \cite{milo2002network}.  Such motifs have been reported to be associated with both single node/edge attributes as well as whole-graph attributes \cite{shen2002network}.  We have here shown that in a manuscript assignment task, the topology around each node is indeed associated with the manuscript source. 
To combine topological information with information propagation, we proposed a novel GCN where the fraction of second neighbors belonging to each class is used as an input, and the class of the node is compared to the softmax output of the node. This structure can be combined with external information about the nodes through a dual input structure, where the projection of the network is combined with the external information. We checked that such a formalism outperforms the current state of the art methods in manuscript classification. Moreover, even in the absence of any information on the text, the network topology is enough to classify the manuscript almost as precisely as by reading the content of the manuscript.
The results presented here are a combination of information propagation and topology-based classification. While each of these two elements were previously reported, their combination into a single coherent GCN based classifier provides a novel content intended method to classify nodes.  With the current ever-increasing concerns about privacy, new content independent methods for node classification are required. The here proposed approach provides the currently optimal approach for such classification tasks. 
